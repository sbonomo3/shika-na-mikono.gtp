\section{Education Sector Competencies}
\subsection*{Technical Competency 1: Effectively teach Tanzanian students}

\subsubsection*{Specific Competency 1.1: PCVs will develop critical thinking, creative thinking, and problem solving skills among students}

Learning Objectives
\begin{enumerate}
\item On the final written technical exam, each PCT will explain the acronym SCALE and list down at least five specific methods for promoting critical thinking, creativity, and / or problem solving skills among students, as described in the technical training manual.

\item Throughout internship teaching, each PCT will prepare at least ten
lesson plans that promote critical thinking, creativity, and problem
solving through the use of the SCALE paradigm, as evidenced by the
Continuous Lesson Preparation and Reflection assignment.

\item  Throughout internship teaching, each PCT will consistently demonstrate use of techniques promoting critical thinking, creativity, and /
or problem solving skills among students, as evidenced by feedback on
the Internship Teaching Evaluation Rubric.
\end{enumerate}

\subsubsection*{Specific Competency 1.2: PCVs will improve subject
content understanding among students to increase their
opportunities for continued education.}

Learning Objectives
\begin{enumerate}
\item On the final written technical exam, each PCT will accurately describe
the structure of the Tanzanian education system, educational pathways
for students, and the nature of formal assessments, as described by the
technical training manual.
\item Throughout internship teaching, each PCT will prepare a scheme of
work and at least ten lesson plans that both specifically address syllabus content and past questions from NECTA exams, according to the
standards of the technical training manual and evidenced through the
Continuous Lesson Preparation and Reflection assignment.
\item Throughout internship teaching, each PCT will consistently demonstrate effective classroom management and gender inclusive strategies as described in the technical training manual and evidenced by the
Internship Teaching Evaluation Rubric.
\item Throughout internship teaching, each PCT will consistently demonstrate use of formal and informal classroom assessments, as described in the technical training manual and evidenced by the Internship Teaching Evaluation Rubric.
\end{enumerate}

\subsubsection*{Specific Competency 1.3: PCVs will promote English
comprehension, communication, and use among students}

Learning Objectives
\begin{enumerate}
\item On the final written technical exam, each PCT will list down at least
three methods for promoting English use among students, as described
in the technical training manual.

\item Throughout internship teaching, each PCT will demonstrate use of
level-appropriate English in classroom teaching, as evidenced by the
Internship Teaching Evaluation Rubric.

\item Throughout internship teaching, each PCT will demonstrate consistent use of English in PCT-student interactions, as evidenced by LCF
observation.
\end{enumerate}

\subsubsection*{Specific Competency 1.4: PCVs will develop life skills among students}

Learning Objectives
\begin{enumerate}
\item On the final written technical exam, each PCT will demonstrate knowledge of HIV/AIDS statistics and list down at least three ways of incorporating HIV/AIDS and/or Life Skill content into classroom teaching,
as described in the technical training manual.
\end{enumerate}

\subsection*{Technical Competency 2: Collaborate with Tanzanian teachers}

\subsubsection*{Specific Competency 2.1: PCVs and Tanzanian teachers
will collaborate to increase their subject specific technical skills}

Learning Objectives
\begin{enumerate}
\item At IST, PCVs will use needs assessment, facilitation skills, and quality
tools, to organize, analyse and discuss data with Tanzanian teachers.
\end{enumerate}

\subsubsection*{Specific Competency 2.2: PCV and TZ teachers will collaborate to increase their capacity to develop lessons
that foster critical thinking, creativity, and problem
solving.}

Learning Objectives
\begin{enumerate}
\item On the final written technical exam, each PCT will describe the process
and content of pre-service teacher training, the current in-service training programs, and the modalities through which Tanzanian teachers
may pursue additional training, as described in the technical training
manual.

\item During micro-teaching, each PCT will demonstrate giving and receiving feedback through the Lesson Study process, as evidenced by LCF
observation.

\item During IST, each PCV will engage in the Lesson Study process with
counterpart teachers according to the parameters of the IST Lesson
Study training session.
\end{enumerate}

\subsubsection*{Specific Competency 2.3: PCVs will promote English
comprehension, communication, and use among Tanzanian teachers}

Learning Objectives
\begin{enumerate}
\item PCV will speak level-appropriate English is a way that clear and understandable to fellow teachers, as evaluated by LCF observation.

\item Throughout internship teaching, each PCT will promote and model the
use of English during interactions with fellow teachers, as evidenced by
LCF observation.

\item PCV will encourage, correct, and support English speaking in Tanzanian teachers, as evaluated by LCF observation.
\end{enumerate}

\subsubsection*{Specific Competency 2.4: PCVs will promote Life Skills,
malaria, and HIV/AIDS awareness among Tanzanian
teachers}

Learning Objectives
\begin{enumerate}
\item At the end of IST, each PCV will be able to informally discuss topics of
HIV, Life Skills, and malaria with fellow teachers in the manner taught
in the IST peer education training.

\item During IST, a PCV will demonstrate active listening skills while recommending testing and treatment options as well as other assistance
for people living with HIV, as evidenced by performance in the Active
Listening Training Role Play exercise.
\end{enumerate}

\subsection*{Technical Competency 3: Engage with Communities}
\subsection*{Specific Competency 3.1: PCVs will involve communities and schools in participatory analysis for community
action}

Learning Objectives
\begin{enumerate}
\item On the final written technical exam, each PCT will accurately describe the structure of the Tanzanian schools and community administration,
as described in the technical training manual.
\item During the session on PACA tools, PCTs will implement PACA tools in
their internship schools, as described in the Peace Corps PACA manual.
\item Before IST, each PCV will use PACA tools to conduct a preliminary
needs assessment in their schools, as evidenced by the School Situational Analysis reporting submitted to APCDs.
\end{enumerate}

\subsubsection*{Specific Competency 3.2: PCVs will work with schools
and communities to develop sustainable teaching and
learning resources}

Learning Objectives
\begin{enumerate}
\item At the end of IST, each PCV will develop with a counterpart a teaching
and learning resource enhancement plan, to the standards described in
the IST technical training materials.

\item During IST, each PCV will learn subject-specific technical skills for
the use of locally available materials in the development of teaching
and learning resources, to the standard described in the IST technical
training materials.

\item Throughout internship teaching, each PCT will consistently incorporate locally available materials as teaching and learning resources, as
evidenced by the Internship Teaching Evaluation Rubric.
\end{enumerate}