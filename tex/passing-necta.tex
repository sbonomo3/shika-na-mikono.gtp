\chapter{Passing NECTA}
\section{NECTA Overview}

     National Examination for Standard 4, Standard 7, Form 2, Form 4, Form 6 and Teacher Training Colleges are prepared by the The National Examination Council of Tanzania.  The NECTA exams are based on the syllabus prepared by Tanzania Institute of Education, which falls under the administration of the Ministry of Education and Vocational Training. The exams are composed by teachers from secondary schools and other experts who are selected by NECTA. Marking schemes are also prepared by teachers and are discussed and agreed upon before the marking of the exam is begun. \\
      There is one simple reality for all students in Tanzania: the national examinations are the only thing that matters.  This can be a source of tremendous frustration for PCVs.  On one hand we want to teach our students well, but on the other hand we are expected to help them pass their exams.  These goals do not have to be in conflict, though it can often feel as though they are.\\   

      As one PCV recounted, ``when I came to Tanzania, I wanted my students to understand chemistry and the world around them...  They did...  And then they all failed their examinations.''  The examinations are difficult because they test obscure material, require memorization of very specific details and each subject has a huge amount of content.  As a result, it is nearly impossible for a student to receive a perfect score on any examination.   \\

      This means that PCVs must consider the examinations in almost every lesson they do with their students.  The exams are beatable in the sense that at some point you can cover all possible material, but this is unlikely to happen in village schools because of language issues, lack of teachers over four years and lack of focus from students.  You will have to condense your material in a way that is easy for the students to understand and memorize. 

\section{Lessons for Passing NECTA}
Here are a few questions to ask yourself when planning a lesson:
\begin{itemize}
\item How was this topic tested on past examinations?
\item What definitions must be given?
\item What drawings must be given?
\item What points must be given? (For essay questions, 5-8 points are required)
\item What topic is it paired with for comparison questions? (Ex digestion and respiration)
\item What common mistakes are students going to make?
\item What is the basic strategy for solving any problem similar to this one?
\item What advantages/disadvantages are there for each subtopic? (Ex advantages of angiosperms)
\item What vocabulary must they know?
\item What past content can I include in this new lesson to help the students review?
\end{itemize}

      Ideally, PCVs will spend a lot of time reviewing past examinations before preparing lessons.  It is good to begin from 2001, when the new format for examinations began.  There are NECTA past paper books available at any book store and past exams are also kept by every school.  Once you have reviewed these past examinations, you will get a feel for what kinds of questions they ask.  When you prepare a lesson, always review past NECTA questions on the topic before planning how you will teach it. \\

\section{Divisions} 
      Divisions are the groups used to determine which students go to A-level.  In order to continue on with education, a student must get division one or two.  It is possible for some division three students to continue on to A-level, especially girls and science students.\\   
      The examination system of Tanzania follows the '3 Cs' rule.  This means that if a student in O-level wants to continue on to A-level, they must receive at least a C in all three subjects in their combination.  For example if they want to continue with the PCM combination, they must have received at least a C in physics, chemistry and mathematics.  Grades in other classes are important in calculating their division.  If a student receives division 1 or 2 with a minimum of 3 C's in their desired A-level combination, they are going to be chosen to attend an A-level school. \\

      If you receive an F in mathematics, English, Swahili, or civics, you will automatically receive division 3 or below.  English and Civics are very easy to pass, so if a student is going to get division 1 or 2, it is unlikely that they will fail either of these classes.  The trouble is with Swahili and mathematics.  Mathematics is a subject that many Tanzanian students struggle with, even though passing (getting 21\%) is possible for even the worst math students.  If you are a math teacher, remember that your students chances of continuing on with their education are greatly influenced by their math grades. \\
      The grades are determined using the following scale (note that a student only needs 21\% to pass any subject):\\

      A 81\%, B 61\%, C 41\%, D 21\%, F 20\% or less\\
      
      The system for determining divisions is simple.  NECTA will take the highest 7 grades and use them to calculate the students division using the following point scale. \\

      A 1 point,  B 2 points, C 3 points, D 4 points, F 5 points \\

      Divisions work like golf, the lower the score, the better the student has done.  Here is a breakdown of divisions by point value (note that the best score is a 7): \\
\begin{flushleft}
Division 1: 7-17 points\\
Division 2: 18-22 points\\
Division 3: 23-25 points\\
Division 4: 26-30 points\\
Division 0: 31+ points, the student has failed the examinations completely\\
\end{flushleft}
 
      In Tanzania, a students future will be determined by the division they get.  This is why it is so important for PCVs to consider the national examinations in everything they do in class.  If you help your students score better on their exams, it can drastically change their options for the future.  The options for various divisions work like this: \\
\begin{flushleft}
Division 1, 2: A-level, lower point totals mean acceptance into better schools\\
Division 3: Some students, especially girls can still go to A-level.  Students taking science combinations have a good chance of continuing if they meet the 3 C minimum, arts students still have a chance though space is limited.  For those who are not accepted, they often end up going to a teachers college to become primary school teachers.\\
Division 4: Police, VETA (vocational schools), nursing, army\\
Division 0: Shamba, machinga\\
\end{flushleft}

\subsection{Form 2 National Exams }
      The Form 2 exams are not particularly important.  They used to prevent students from continuing on the form 3 and 4, but this has since ended.  This means that students do not have any meaningful examinations until their form 4 NECTAs. This can cause problems for PCVs who teach chemistry and physics.  In Form 1 and 2, these subjects are mandatory, but they can be dropped during form 3 and 4.  This means that form 1 and 2 students who know that they will drop these science classes often don't pay attention in class. \\

      There are no divisions for form 2 students.  Instead they just use averages.  The grades are calculated as follows:\\

      A 81\%, B 61\%, C 41\%, D 30\%, F 29\% or below\\
       
\section{Exam Formats and Grading} 

      All examinations for subjects taught by PCVs involve some combination of multiple choice, fill in the blank, matching, essay and math problems.  The grading for much of this is obvious and does not need explanation, however, the math problems and essays do require a bit of explaining.  The information below is targeted towards English and mathematics, but it applies to similar questions in the other subjects as well. 
      
\subsection{NECTA Math Grading} 
      Grading mathematics is very simple.  The number of points a question is worth is generally related to the number of steps involved in solving a problem.  The final answer is only worth 1 point and the rest is split among the work shown to arrive at the answer.  Students must write all of their answers neatly, showing each step clearly for the grader to grade.  Scratch work is written on a separate paper and is not turned in with the exam.  During the national exams, the right side of the paper is used for marking only.  It is a good idea for students to use a ruler to make this space when the are doing exams in school prior to taking the national examinations. This way they get used to leaving this space to the graders.  Writing in this area will result in lower scores. 

      When students are writing on their answer sheets, they do not have to write their answers in order.  It is expected that their answers are clearly marked and written neatly.  As with other classes, all drawings are expected to be drawn in pencil, while everything else is written in ink.  Students are allowed to skip steps in algebraic simplification if they are able to do it correctly. 
\begin{flushleft}
    
\textbf{Section A (60 Marks) }\\
Contains 10 questions worth 6 points each\\
Questions come from any topic from form 1-4\\
 
\textbf{Section B (40 Marks)} \\[20pt]

Contains 6 questions, the student must choose 4 and they are worth 10 points each\\
Topics are only from form 3-4 and the topics are always predictable\\[20pt]

Q11 is linear programming\\
Q12 is statistics\\
Q13 is 3 dimensional shapes and earth as a sphere\\
Q14 is matrices and transformations\\
Q15 is book keeping and accounting\\
Q16 is probability and functions/relations\\[20pt]

Since the questions come from predictable topics, its good to review the easier topics like linear programming, statistics and earth as a sphere with form 4's before their exams.  Since each question is worth 10 points, getting 2 of these correct will almost guarantee that the student passes mathematics\\

\end{flushleft}  
 
\subsection{NECTA Essay Grading} 
  For many new PCVs it can be a bit of a shock to watch a Tanzanian grade an essay.  They grade it in less than thirty seconds and pay little attention to the grammar or spelling.  There is little thought put towards the quality of the argument being expressed.  What is most important is that the essay is easy to mark and that the examples listed are found in the grading scheme.   

      Essays require between 5-8 points supporting their answer.  In the past 5 points were required, but during the 2010 examinations 8 points were required.  This was not announced by NECTA and it is unclear if this will continue.  It is a good idea for teachers to try to give 8 points to their students instead of 5, in case this continues in future examinations. \\

      When grading essays there are a few general guidelines to follow.  All essays are worth 20 points and the grading generally breaks down like this: \\

\subsubsection{Introduction (2 points):}
\begin{itemize}
\item The introduction must contain the definition of the topic being discussed
\item It should contain key participants, theories, or people that are relevant to the topic
\item If it is a question asking for an opinion, they should state very clearly what their opinion is
\item If they are writing an essay where they must imagine that they are someone, it should be done in the voice of that person.  For example if the question asks for a student to write an essay as an Agricultural Extension Officer, its best to begin with something like Dear people, I am in front of you today as your Agricultural Extension Officer to give you advice on how to improve farm yields.
\end{itemize}

\subsubsection{Body (15-16 points)}
\begin{itemize}
\item Contains 5 or 8 points which support their view.  The best points should be written at the beginning, in case they are only grading 5 points.  Its best to try to write 8 unless you know that NECTA will grade only 5.
\item Each point must be its own paragraph.  Most paragraphs consist of only 2-3 sentences.
\item 1 point is given for the point itself.  It should be clear and simple.  Often students underline the point, though this isn't recommended or required.  The first sentence of each body paragraph should just be the point stated with nothing else.  The support comes in the second and third sentences
\item 1 point is given for the explanation and examples found in the second and third sentences.  This can also be given for neat writing and correct paragraph arrangement.  It depends on the grader of the examination.  If it is a 5 point examination, this part is worth 2 points instead of 1, usually split between neatness and support of the point being made.
\item Grammar is not considered unless there are very serious errors or the sentence cannot be understood, even for English exams.  NECTA graders often do not have good enough English to correct these essays.  Also, there are so many errors made by students that grades would be severely affected if grammar and writing quality were considered.
\item Flow is not important.  Students are not expected to use transitional sentences.
\end{itemize}

 
\subsubsection{Conclusion (2-3 points)}
\begin{itemize}
\item Contains a short summary of what you have written
\item Contains the students opinion on the issue of the essay
\item They can begin the conclusion with In my view... or In summary...
\end{itemize}

Some of these criteria might make an English teacher squirm, but remember that the most important thing is that the essay is easy for the grader to mark.  This format does not exist because the Ministry of Education thinks that this is the ideal way to express ones ideas, but rather it allows them to push through thousands of examinations in a reasonable amount of time.