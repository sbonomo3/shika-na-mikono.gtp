\chapter{Common Challenges and Possible Solutions}
We will face a variety of challenges while teaching in Tanzania.  Below are a list of issues volunteers regularly face in their school and ways other volunteers have overcome them. 

\section{Lack of Teachers, especially in the sciences}
Unfortunately, teacher shortage is the most common issue plaguing our school and one we can do little about.  However, there are ways we can compensate for the lack of teachers.

\begin{itemize} 
 \item Try establishing a ``reading room'' in your school with the textbooks you have on hand.  Students can come here to study when a teacher is not present. 
 \item Provide study guides for classes with no teachers.
 \item Choose a student from a higher form to teach forms with no instructor.
 \item Elect student leaders to help with administrative tasks.
\end{itemize}

\section{Poor educational background in students}
Students may enter your school without the ability to read, write, or perform basic mathematical calculations.  Try these tips to overcome poor primary education.

\begin{itemize}
 \item Always teach new Form I students.  Speak to them in English and make learning the language fun for them, so they will continue to study it after the initial orientation course is over. \
 \item Use the first day of a new topic to review material that should have been learned in the past. 
 \item Have academic camps during school holidays to encourage students to re-learn the basics. Math camps have been extremely successful for a number of volunteers.
 \item Create monthly academic competitions between forms, like spelling bees or trivia bowls and offer cash rewards.
 \item Create remedial classes for students who are slower learners.  It allows the teacher to focus on a student's weakness and ways to improve.
 \item Offer free one-on-one or small group tuition. Post your schedule for students to sign up for an hour include name, subject and topic so you can prepare and meet with them in their classrooms.
\end{itemize}

\section{Language barrier}

Its true most of our students don't know English.  But they aren't going to learn it if you teach them in Swahili either.  Try fun and interactive lesson plans that can allow you to teach in English and still convey meaning.\\

See Chapter 7 or the English Teaching Manual for more ideas.

\section{Corporal punishment}
When schools are  hard pressed to find teachers, keeping discipline in a school becomes difficult.  This is why so many schools use corporal punishment; it is quicker to hit a student with a stick than stay after school to administer detention.  Other that the obvious human rights violations, corporal punishment does not fix the core problem: students' behavior. Students fearing a stick means they will only behave when threatened with punishment.  Therefore, when we use corporal punishment we are not teaching our students how to behave and this will result in a disconnect between actions and results for the rest of their life.  Below are the national government regulations regarding corporal punishment:
\begin{enumerate}
\item This regulation may be cited as the National Education Corporal Punishment Regulation, 2002.
\item In these regulations, unless the context otherwise requires: Corporal Punishment means punishment by striking a pupil on his hand or on his normally clothes buttock with a light, flexible stick but excludes striking a child with any other instrument on any other part of the body.
\item Corporal punishment may be administered for serious breaches of school discipline or for grave offenses committed whether inside or outside the school which are deemed by the Head of School to have brought or are capable of bringing the school into disrepute.
\item Corporal punishment shall be reasonable having regard to the gravity of the offense, the age, the sex, and the health of the pupil and shall not exceed 4 strokes on any one occasion.
\item The Head of School in his discretion may himself administer corporal punishment or delegate his authority in writing to all or any member of his staff provided that the member or staff authorized may only act with the Head of School on each occasion when corporal punishment is administered.
\item Female pupils may only receive corporal punishment from female teachers, except where there is no female teacher at the school in which case the Head of School may authorize in writing a male teachers to administer corporal punishment or may himself administer such punishment.
\item All occasions on which corporal punishment is administered shall be recorded in writing in a book kept for the purpose and such record shall state in each instance the name of the pupil, the offense or breach of discipline, the number of strokes and the name of the teacher who administered the punishment. All entries in this book shall be signed by the Head of School.
\item Refusal to accept corporal punishment either by a pupil or by the parent on the pupil's behalf may lead to the exclusion of the pupil in terms of expulsion and exclusion of pupils from schools' regulations made under the provisions of the National Education Act, 2001.
\end{enumerate}

The best way to overcome corporal punishment at your school is to encourage and reward those with good behavior.  For students with poor behavior, you may refer to Chapter 5 for discipline measures and also hold counseling sessions with the students.  Some may be having problems at home or struggling with academic material. Additionally, many bad behaving students are also truants. If a student misses 90 days of school, they are expelled.  Keep track of students attendance and remove those students who do not want to study and are taking away learning opportunities from their peers.
 
\section{Lack of commitment from teachers}
Talk to the teachers at your school to find the cause of their lack of
commitment.  Be realistically optimistic and try to support them in
ways you can.  Avoid being negative.  Share successes and challenges
and encourage other teachers to do the same.

\section{Limited resources}
Use the subject specific manuals to find alternative forms of teaching materials. 

\section{No student support system after school or at home}
Students live at home, in hostels or in ``ghettos".  A ``ghetto" is a
room rented by students whose families live in neighboring villages.
During the school year some students have no support at all.
\begin{itemize}
\item Stimulate parent and community involvement by hosting an event for
them (open house, drama, or talent show).
\item When parents or other villagers are complaining about the state of
local education, explain the roles of PTAs and school boards in
America and how much control parents and community could have in
school systems.
\item Ask students what they do after school, visit their ghettos and host
extracurricular activities or tuition.
\item Start a life skills, peer education or big brother/big sister
program to keep students from negative influences and to give them
confidence and experience to support each other.
\item Take students up on invitations to visit their families.  Our
conversations with parents and relatives about education, America,
coming to Tanzania might inspire a family to support their children in
education.
\item Encourage students to become supportive parents.  Emphasize the
importance of education and the impact of teaching children at a young
age.  Maybe one day students will remember our preaching and their
children will be supported.
\end{itemize}

\section{Lack of student interest to learn}
Teach the students who are interested.  Spend time out of class
getting to know the others.  Most of them do not understand the power
of education.  Some have difficult situations at home, others have
insufficient academic background.
\begin{itemize}
\item At the beginning of each semester teach study skills (note taking, test taking, how to have a group discussion, etc).
\item Provide positive alternatives to academics. Participation in sports, drama clubs, self-reliance projects and life skills programs gives students skills they will be able to use when they have failed school.
\end{itemize}
\section{Teacher training}
Teachers with university degrees tend to get jobs in A Level schools,
Teachers' Colleges or administrative positions.  Other teachers have
certificates to teach in primary school or diplomas to teach secondary
school from 2 years of post-secondary study in Teachers' Colleges.
Part of the education in Teachers' College includes "field" a student
teaching experience in village school.  Schools with insufficient
teachers will hire students who have finished form 6 with Division
3-15 or higher.  Form 6 leavers have possibilities of continuing their
studies and usually work short contracts with the Head of School.\\

There are many opportunities for professional development at work.
Take time to get to know and understand  your colleagues.  Encourage
feedback from your fellow teachers.  Likewise, if you see a weakness,
find an appropriate time and place to offer advice on how to overcome
that weakness.\\

Respect other teachers, even those who have little training or those
with whom you disagree.  It is easy to think of teachers (especially
form six leavers) as incapable of teaching but better to think of them
as potential.  Sharing alternative teaching methods and successes in
the classroom may inspire you and others and make your experience much
more rewarding.

\section{High school fees}
Try setting up a scholarship fund from donors back home. Or reach out to local NGOs who fund education.

\section{Poor School Leadership}
In some instances, your Head of School or other administrators may not be fulfilling their duties. Thankfully, many of us teach in schools with a small teaching staff, which can allow us to enter roles of leadership and compensate for these shortcomings.  Additionally, Peace Corps has been trying to foster relationships between District Education Officers and PCVs.  If your Head of School is not listening to your point of view or violating their role as Head of School, you may seek advice from fellow teachers and possibly consult your DEO.
