\chapter{Classroom Management}
Good classroom management is essential to creating a learning environment for your students and allowing you to love your job.  In order for our students to feel comfortable learning, we need to establish an environment that is conducive to engaging all types of students.  The teacher as a guide and facilitator helps to establish a learning environment of collaboration and cooperation.

\section{Manage your space}
In most schools, teachers will be entering a classroom for each form, rather than having their own space like in the United States.  This does not mean you can't arrange the classroom or make it your own during your class period.   

\begin{itemize}
 \item Set up classroom rules with your students. Usually 5 positive rules are sufficient. For example: We will not laugh at other students, Students will be prepared for class, Students will complete assignment, Students will be on time.  Have consequences clearly defined if these rules are broken and stick to them.  
 \item Walk around the classroom while lecturing to monitor behavior and understanding in your students.  
 \item Use non-verbal signs with students to indicate being quiet or going to the bathroom.
 \item Divide your board into 3 or 4 sections at the beginning of class.  Use 2 or 3 for notes, leaving the final space open for pictures, examples, or vocabulary building. Make sure to write the date and subject in the top right corner of the board.
 \item Do not try to talk while students are still writing notes. Instead, have students put their pens down or in their pockets when you want them to listen, so they can not write while you teach.
 \item Greet your students when you enter class and thank them for a good lesson at the end of class.
 \item Teach subject content on the first day of school.
 \item Have students create posters to hang around the class to help with their English or other subject matter.  Create a ``clothes line'' and hang the posters with clothes pins so they can easily be removed on test days.
\end{itemize}

\section{Know your students}
It is important  that we know our students, both to encourage them in the classroom and to know when they may need our help in other areas of life.  

\begin{itemize}
 \item Learn students names.  A good English exercise for this is having students stand in a circle.  The student says ``This is (the name of the person to the left). My name is \_\_\_\_'' and continue around the circle in that manner. An occasional roll call, being observant
when handing back papers, and mentally registering names while
students are writing tests.  Students will be surprised and proud if
you call them by name unexpectedly.
 \item Set up office hours.
 \item Have meetings with students when you hand back tests to talk about strengths and weaknesses. 
 \item Local and national culture will greatly influence your students behavior.  Be sensitive to this fact.
\end{itemize}

\section{Expect the best}
We need to have high expectations for our students and believe they can succeed.  If we have high expectations for our students, they will try to meet them. 

\begin{itemize}
 \item Say the learning objective at the beginning of the class.  ``Today, we are going to learn how to write an essay.   By the end of class, you will be able to write an essay on HIV.''
 \item Make sure your students are writing notes and not working on other subjects during your period.  
 \item Call on students who may have the answer but are afraid to speak up.  Give homework and have students share in the following class.  
 \item Allow students to teach class or serve as an example when they do an assignment especially well.
\end{itemize}

\section{Establish Procedure}

Create routine in the classroom so students know what to expect.  This saves time and your voice from having to repeat instructions.   Consider things like how to begin class, bathroom breaks, group work, writing notes, listening to instruction, answering questions, and students who may be off task or unprepared. Use the following classroom procedures or create your own. 

\begin{itemize}
 \item Greet students and ask them to sit in their seats.
 \item Everyone gets out the materials they need for class.
 \item When the teacher is writing, students are writing.
 \item When the teacher says ``Pens down", students are listening.
 \item To use the bathroom, a student must raise their hand with fingers crossed.  Teacher will nod to give permission.  They will re-enter without interrupting class.
 \item When answering questions, students will stand straight and speak loudly and clearly.
 \item Teacher will walk around the room to monitor class progress and understanding.  Students not on task or unprepared for class will be asked to leave.
 \item At the end of class, students will ask questions and write down homework assignment.
\end{itemize}

\section{Give praise}
Many of our students are coming from homes where they may receive little vocal praise or affirmation.  Be sure to reward good behavior and work ethic.

\begin{itemize}
 \item Show off students work in the school library, class, or school display board.
 \item Use positive words of praise.  (Good, Great, Awesome, Good Job, Very cool, Well Done)
 \item Allow getting the wrong answer to be okay. ``Good idea, but I asked \_\_\_\_\_. Can you try again?'' ``Think, are you sure that is correct?''  
 \item Reward high scoring students and most improved students with pencils, stickers, or candy.
\end{itemize}

\section{Disruptive Behavior}
Hopefully, you will establish a classroom that will have few discipline problems.  However, there are times when students may break rules.  The following are ideas for preventing problem behavior and what to do when it may occur.

\begin{itemize}
 \item Establish working rules and consequences.
 \item Establish a relationship with students based on respect.
 \item Give praise for appropriate behavior.
 \item Encourage students to work together.
 \item Involve parents in their child's education and behavior.
\end{itemize}

Unfortunately, many of our schools are understaffed, creating meaningful and behavior changing consequences for misbehavior can be difficult.  Remember, punishment and consequences are two different things.  We may not hit our students, but having them kneel in the front of class or do push-ups doesn't teach them why what they did is wrong.\\  

Try the following consequences for common problem behavior:
\begin{itemize}
 \item Put on a serious face and stare at the student until they become quiet. Ask them what they are doing.
 \item Ask a misbehaving student what they are doing and what they should be doing right now.
 \item When a student asks to be forgiven tell them you forgive them, but because they have broken a rule, they will face the consequences.
 \item Cut off a student if they continue to argue. ''We will discuss this at another time, now you need to \_\_\_\_"
\end{itemize}

There are times when you may be one of two teachers at your school.  During these times it may be very difficult to execute the above-mentioned discipline measures.  Do not let other students learning suffer because one student's poor decisions.  If needed, remove the disruptive student from class and deal with them later.  Show your students that it is a privilege to study in school and one they should not take for granted.


\section{Get back your classroom}

Sometimes we can have a rough day in class that can make us doubt ourselves, which affects our teaching and classroom management.  If you feel that student behavior is slipping, follow these guidelines to get back on track:

\begin{itemize}
 \item Relax and get a good night's sleep
 \item Review the rules you set for the class.  Why are the students violating them?  Do the students understand them?
 \item When you enter the classroom, be firm but positive.
 \item Review the rules again with your students, clearly stating the consequences if broken.  Be confident.  You are a good teacher and deserve to be treated as such.
 \item Be firm and consistent in enforcing the rules.  Have a zero-tolerance policy for rule breaking.
 \item Seek advice from a counterpart or another Peace Corps teacher. Remember, you are not alone! 
\end{itemize}
