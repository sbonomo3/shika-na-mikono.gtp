\section{Biology}
\subsection{Form I}
\subsubsection*{Introduction to Biology}
\begin{itemize}
\item	When teaching fields related to biology, draw picture flashcards, quizzing students on meaning.
\item	Ask students what they think they can use biology for and have them make a short list that can be shared with the rest of the class.
\item	If they are totally ``clueless'' try to bring  in one or two professionals who studied biology as preparation for their career.
\item	Have a plant, human, and insect.  Ask students how we know these things are living.
\item	Ask students to use 5 senses to verbally describe a flame. Use this experiment as example for lab notebook
\item	In groups of 4 students will conduct a series of experiments (Who is the tallest in your group? Who has the highest temperature? Who has the quickest pulse? What is heaviest, rock, pen, or shoe?) using scientific method and lab notebooks to record.
\end{itemize}

\subsubsection{Safety in Our Environment}
\begin{itemize}
\item	Draw or have students draw pictures of common household accidents.
\item	Have students draw the common hazard signs in a laboratory and hang around the classroom..
\item	Act out First Aid procedures in class, making sure students know the name of the injury and how to treat it.
\item	Bring in a First Aid box.  Ask students what is inside and what it is used for.
\end{itemize}

\subsubsection{Health and Immunity}
\begin{itemize}
\item	For manners and hygiene: have students draw pictures of people with good manners and people with bad manners. Same with good hygiene and bad hygiene.
\item Use chalk dust or glitter (beauty shops) to show germs and how easily they can be passed from once person to another. 
\item Have four students link arms to create a wall, showing the body's first line of defence. Have another student, or pathogen, try to break through the line to enter the blood stream.  Then remove one of the students from the first line of defence and have the pathogen enter through that hole.  Two other students will act as white blood cells to try and fight the incoming pathogen.  Be sure students have name cards or sheets of paper to indicate their role in the play.
\item Use activities from the Life Skills Manual, HIV manual, or MoEVT peer teaching book when giving lessons about HIV.
\item Role play ways to say no to sex.
\end{itemize}

\subsubsection{Cell Structure and Organization}
\begin{itemize}
\item Teacher draws the two cells on the board and has students try to place labels on the correct parts of the cells.  Students learned parts of cell in primary school, but in Kiswahili, so this activity will give them confidence to show what they already know.
\item Have students observe onion cells using a water drop microscope. (google water drop microscope)
\item Give 8 students a manilla card, each saying a different level of organization (atoms, molecules, organells, cells, tissue, organs, organ systems, organism) Have the class try to arrange the students from smallest to largest to show order of organization.
\end{itemize}

\subsubsection{Classification of Living Things}
\begin{itemize}
\item	Give students cards which they will draw plants and animals they see around them.  Then have them classify based on similarities and differences.
\item Using playing cards, have students group the cards by color (not as specific) and then by suit. (more specific) 
\end{itemize}

\subsection{Form II}
\subsubsection{Classification of Living Things}
\begin{itemize}
\item Bring in yeast, mushroom, and breadmold for students to observe with a magnifying glass and draw.
\item Start the class with a nature walk oberserving different plants. Ask students what the plants all have in common, leading to the features of Kingdom Plantae.
\end{itemize}

\subsubsection{Nutrition}
\begin{itemize}
\item Begin the class by asking students to call out their favorite foods.  At the end of class, use these foods as examples to classify as carbohydrate, protein, lipid, ect.
\item Show students pictures of different organisms and have them call out if they are heterotropic or autotrophic.
\item Have students create a menu for the day that is a balanced diet. They must include 6 serving of grain, 3 of milk, 3 of protien, 2 of vegetables, and 2 of fruit.  
\item Draw the digestive system on the board and have students guess the location of certain organs as you teach them using card labels.  You can also write the function of the organ on the back of the cards and review the next lesson reading the function and students calling out the organ.
\item Draw 8 plants on different pieces of paper, one normal and the others suffering from different mineral deficiencies. Have students observe each and compare to the health plant picture.  Ask them what they observe, using their observation to lecture on mineral difficiencies in plants. Quiz students later with the pictures and ask them what mineral is missing and how they know.  
\item	For Photosynthesis, have students draw a real leaf to learn the parts.
\item Remember to do photosynthesis and food test practicals, they are simple and kids love them.
\item Bring in different foods that have been preserved and have students try to guess which method of food preservation was used. Then have students taste a food that was preserved by drying and the same food fresh.  Ask them if they taste the same or different. Explain that this is a negative effect of traditional methods, they change the taste of food.
\end{itemize}

\subsubsection{Balance of Nature}
\begin{itemize}
\item After lecturing on the natural environment, have students race to complete a scavenger hunt.  Write 5 questions on 5 different slips of paper.  When they finish the task of one question, they can get another task. The first group to finish wins. Example questions: draw an example of a secondary consumer, bring the teacher an  abiotic thing in the environment, write the definition of an ecosystem, draw an example of a producer, list four reasons we should protect the environment.
\item Break students into small groups and give them pictures of different organisms and a piece of chalk.  Have students first group the organisms by their tropic level, then create a food chain, and finally a food web.  Walk around and observe each group. 
\end{itemize}

\subsubsection{Transport of Materials}
\begin{itemize}
\item Give a few students pieces of paper that say waste, energy.  Have two students acts at blood cell, one handing out cards that say food and oxygen and the other that collects waste and energy.  Explain that our blood transports materials.
\item Spray perfume in the corner of the room and have students raise their hand when they can smell it.  Then have students repeat the exercise, using themselve to represent the particles, starting in one corner of the room and dispersing until they are evenly distributed.
\item Using desks to represent a membrane, have students diffuse from one side to another.  Then show a semi permeable membrane and only let girls pass through.
\item Using coloured chalk, draw a large heart and the rest of the circulatory system on the floor.  Have students walk through the circulatory system, stating where they are and where they are going next.
\item Place a few drops of Gentile Violet (purple stain used for surgery, Pharmacy) into water and then place a cut stem of a monocot and dicot into the solution.  After a few hours, make a thin cross section of each and have students observe using a magnifying glass.  They will be able to see the vascular bundles and differentiate between monocots and dicot. 
\end{itemize}

\subsubsection{Gaseous Exchange and Respiration}
\begin{itemize}
\item Have students breath in through their nose only.  Then have them breath through their mouth only.  Compare the two types of breathing and list the benefits of breathing through the nose.
\item Show photos or have students observe different types of organisms and the organs they use for respiration.
\end{itemize}

\subsection{Form III}
\subsubsection{Classification of Living Things}
\begin{itemize}
\item After presenting information about moncots and dicot, give students specimens labeled A, B, C, and D (maize, bean, mango, and cashew) Have them identify kingdom, phylum, class, common name, and two observable features that allowed them to classify the specimen.
\end{itemize}

\subsubsection{Movement}
\begin{itemize}
\item Teacher acts out each form of locomotion and students call out what type. For ciliary use your fingers, for flagellary use a belt as a flagellum, and for muscular, walk.
\item Bring in joints from the butcher to show students different types.
\item Have students grip their bicep, flexing and relaxing their arm to feel the shortening and lengthening of the muscle.
\end{itemize}

\subsubsection{Coordination}
\begin{itemize}
\item	Throw a ball at a student.  Have them then list the steps of nervous coordination from stimulus to response.
\item Break students into small groups. Give each card that says a stimulus on it and have students write the 5 components of nervous coordination that will occur. For example: Seeing a snake. Stimulus: Snake, Receptor: Eyes, Coordinator: Brain and spinal cord, Effector: Leg muscles, Response: Run away! 
\item Have many students stand in a Y shape. One end of the Y is sensory neurons which lead to the brain and then go back down to the other end of the Y which is motor neurons.  Have students send a message (either squeezing hands or saying a word) that starts at the sensory neurons travels up the relay neurons to the brain and then back down to the motor neurons.
\item Measure students reaction time by dropping a meter stick through their open hands. Explain that this is nervous coordination.
\item Draw a large nerve ending, synapse, and dendrite of another nerve on the ground.  Have students act as synaptic transmitters and cross the synapse to send a message.
\item Use dominoes to show threshold, refractory, and all-or-none principle.
\item Role play effects of drugs and ways to say no.
\end{itemize}

\subsubsection{Excretion}
\begin{itemize}
\item Draw a large nephron on the ground with chalk that is big enough for students to walk through. Split the students into groups, giving each group pieces of paper with  the name of different substances on it (water, salt, sugar, blood cells, urea). One by one have the groups walk through the nephron, with different substances either leaving the nephron to go back to the blood or following the nephron all they way through to the bladder. Have two students be sphincter muscles which let the other students out of the bladder.
After all groups have gone, allow students to go one by one, each time getting a different substance and leaving the nephron at a different places. If time allows, you can also introduce the role of ADH by having some students labeled ADH go in and pull water back out of the nephron and into the blood stream. 
\end{itemize}

\subsubsection{Reproduction}
\begin{itemize}
\item Have students match pictures of asexual reproduction with the different methods.
\item Send students outside to collect any flower they like, encourage them to find one that is unique and different. Then have students draw and label their flowers, determining whether they are male, female, or bisexual.
\item Ask students to list all the changes that have occurred in their bodies between Form I and Form III. Explain that these changes are puberty!
\item Draw a large uterus/vagina and penis on the ground. Put the students into groups and give each group a different method of contraception (pills, condom, IUD, spermicide...). In each group, some students will act as sperm, another as and egg and other as the birth control method. Have the class count down (3-2-1 EJACULATE) and the sperms run out of the testes and try to get to the eggs. The contraceptive students prevent the pregnancy from occurring. For example, a condom would create a barrier to prevent the sperm from entering the vagina, an IUD would allow the sperm to meet the egg but then push the egg out so it cant implant, pills would take the egg away, tubal litigation would cut the path so the egg cant pass and spermicide would kill the sperm.
\item Mitosis and meiosis can be acted out in a song and dance. Then students can remember the stages in the song form and the actions that go along with each stage.
\end{itemize}

\subsubsection{Regulation}
\begin{itemize}
\item Ask students to act as if they are very cold.  Note what they do (shiver, rub their arms, etc.) Then have students behave as if they were in a very hot environment, and note their behaviour (sweat, fan themselves, etc.) 
\end{itemize}

\subsection{Form IV}
\subsubsection{Growth}
\begin{itemize}
\item	Plant bean seeds in water bottles and stagger them so you can show the class stages of growth or have students grow their own.
\item	Using matches, string, and paper circles, students act out the stages of Mitosis
\item	Choose 9 students, 2 are DNA, 1 is mRNA, 1 is a ribosome, and 5 are amino acids. Draw a circle on the ground to represent the nucleus, where the DNA will stand.  Each amino acid has a number, 1-5 on them.  DNA will create a combination of the numbers 1-5 and write on a piece of paper.  The mRNA enters the nucleus and makes a copy of this combination and carries it to the ribosome.  The ribosome then arranges the amino acids accordingly.  Protein Synthesis!
\end{itemize}

\subsubsection{Genetics}
\begin{itemize}
\item Using post-it notes, write different bases (A,T,G,C) on a few.  Draw two lines on the board and add a few bases to the left strand.  Have students place corresponding bases on the other strand and then write the genetic code for that DNA strand.  These post-its can be used later to show different types of mutations.
\item Have students race to complete Punnet Square questions.
\item Have students measure their height, hand span, number of boy and girls, and those who can and can not roll their tongue.  Graph each to show continuous and discontinuous variation.
\item Break students into groups. Each group will get two piles of seeds. The first pile will be all maize seeds (X chromosomes)-this the mother-and the second will be half maize and have beans (half X and half Y)-this is the father. Students should close their eyes and draw one seed from each pile. If the get two maize seeds, it is a baby girl. An X and a Y is a baby boy. This can be extending with another trait like albinism or eye colour where a dominant and a recessive trait is involved.
Similarly this activity can be done with coins which have taped letter (X and Y or B and b) on either side. Students should choose from the pile or flip the coins many times and then record the percentage of each phenotype and genotype observed. The teacher can compile the data from the whole class.
\item For variation, have two students stand in front and have the class give all of the differences between them. Then sort these differences into inherited or acquired differences and continuous or discontinuous.
\end{itemize}

\subsubsection{Classification of Living Things}
\begin{itemize}
\item To begin topic, have students race to write as many animals as possible in 3 minutes.
\item	Using hand drawn pictures of organisms on manila paper, teacher uses them like flashcards, quizzing students on kingdom, phylum, class, and three features of each.
\item	Students ask 20 questions to get to an organism the teacher is thinking of.  For example, is it in Kingdom Animalia? Does it have wings? Is it warm blooded? 
\item Have students create Public Service Announcement on how to prevent getting Tapeworm.
\item Have students complete a Venn Diagram showing the differences between Nematoda and Annelida.
\item Show students an x-ray or skeleton to prove that humans do have a tail and belong to Phylum Chordata.
\item Break students into small groups.  Each group draws a different mammal (bat, cow, human, dog, lion, goat, cat) Ask students to call out what they see in their picture and write these features on the board.  Circle the common features that all organisms share (teeth, hair, mammary glands).
\end{itemize}

\subsubsection{Evolution}
\begin{itemize}
\item Have students come together and then explode apart to show the Big Bang Theory.
\item Show students a fossil of a snake when they had legs (internet) then ask students if snakes have legs now. 
\end{itemize}

\subsubsection{HIV, AIDS, and STDs}
\begin{itemize}
\item Use lessons from the Life Skills Manual, articles from FEMA, and guest speakers to teach students about HIV.
\end{itemize}
%===================================================================================================
\section{Chemistry}
\subsection{Form I}
\subsubsection{First Aid}
\begin{itemize}
\item	Role play injuries and have students treat each other.
\item	Build first aid kits with locally available resources
\end{itemize}

\subsubsection{Chemistry Laboratory}
\begin{itemize}
\item	Have students write the name of lab apparati on their back, then ask other students yes or no questions to figure out what apparatus they are. “Am I made of glass? Do I hold liquid?”
\item Draw a picture of the set up on the board and pass out pieces of paper with the names of all the apparatus. Have students place their label on the apparatus in the picture. 
\end{itemize}

\subsubsection{Fire Fighting}
\begin{itemize}
\item	Students can collect different fire starting material and then sort.
\item Teacher pretend to light self on fire/class on fire using different materials (i.e. first ``light'' clothes on fire, then spill ``petrol'' then have an electric fire.) The fire can be a piece of paper with tape on it that is coloured and says ``FIRE!''. The students should use materials in the class (water, sand, blanket) to ``put out the fire'' and save the teacher.
\end{itemize}

\subsubsection{Scientific Method}
\begin{itemize}
\item	For observation, light a ``candle'' that is actually made of a potato and a nut for the wick. Have students write observations about the candle and then eat it - throw them for a loop. Discuss difference between observation and inference.
\end{itemize}

\subsubsection{States of Matter}
\begin{itemize}
\item	Bring in water, a rock, and a balloon and ask students to classify each into a state.
\item	Write states of matter on the board.  Then hand students cards reading “evaporation, condensation”.  Students place cards where they see fit, other students correct as needed.
\item	Have students lock arms to try and represent the different states of matter. Solid is when the students have interlocked elbows, liquid is when they are holding hands, and gas is no holding at all.
\item	Alternatively, have students act out solid liquid and gas by standing close together (solid), walking around in a small area (liquid) and running around (gas). This might be best as an outside activity. Students one by one leaving the liquid state and running around outside of it as vapour can to show evaporation.
\end{itemize}

\subsection{Form II}
\subsubsection{The Periodic Table}
\begin{itemize}
\item	Teacher draws blank periodic table on the board.  Students are given elements and try to place in the table.
\item	“Build” atoms on the desk with 2 different color beans and corn for protons, neutrons, and electrons. Show different isotopes.
\item	Make cards with the information about the first 20 elements (mass number, appearance, s/l/g, etc.) and have students arrange them according to mass and properties. Guide them to group them into columns with similar properties. They should arrange the elements in the same way as Mendel did.  
\end{itemize}

\subsubsection{Water}
\begin{itemize}
\item	Write the word water in the middle of the board and have students brainstorm everything they know about water. Make a giant mind map. Alternatively have students make mind maps in groups.
\item Have different parts of the room labeled as the different natural sources of water (rain, river, lake, ocean, cloud, glacier...) at each place have a die which can be rolled to tell a student where to go next (some sides of the die will say ``stay''-for example for the ocean 4 sides of the die will say stay and two sides will say ``clouds''). Have students go around the class from place to place and record each place they went. After the activity, have students write stories about a ``day in the life of water molecule''. Where did they go and what did they do?
\end{itemize}

\subsubsection{Oxygen / Hydrogen}
\begin{itemize}
\item Have students draw pictures to represent the different uses of oxygen and hydrogen.
\item Set up or draw the apparatus needed for production of hydrogen and oxygen but do it with many mistakes (e.g. have the gas jar upside down, don't put the water in the trough, forget to attach the delivery tube...). Have students correct all of the mistakes so that the set up is correct.

\end{itemize}

\subsubsection{Oxidative States}
\begin{itemize}
\item	If potassium permanganate is available a dilute solution can be made. Add some sodium hydroxide and put a drop on a piece of filter paper. The color will change from purple to green to brown to pink/colorless as the manganese is reduced by the cellulose in the paper. Explain that the different colors are caused by the different oxidation states of manganese.
\end{itemize}

\subsubsection{Nomenclature}
\begin{itemize}
\item	Assign students elements, and then write the name of the molecule on the board, The students then must try to find the right combination of ``atoms'' to form the molecule. 
\end{itemize}

\subsubsection{Electrovalent Bonding}
\begin{itemize}
\item	Have a boy(+ metal) and girl(- non metal) act as an ionic bond, explaining its like marriage, the boy gives a ring (charge) to the girl and they are linked.  Then have two girls (halogens) act as covalent bond, sharing and supporting each other equally as friends.
\end{itemize}

\subsection{Form III}
\subsubsection{Introduction to Reactions}
\begin{itemize}
\item	Give demonstrations of all the different reactions as they are introduced: double decomposition (ppt between  Na2CO3 and CuSO4), replacement (steel wool dipped in CuSO4), combustion (burn paper), combination (heat copper and sulphur), decomposition (H2O2=H2O+O2) 
\end{itemize}

\subsubsection{Balancing Equations}
\begin{itemize}
\item Have students act out the molecules in a reaction. For example, for the decomposition of hydrogen peroxide, have two student be hydrogen and two be oxygen in hydrogen peroxide. Then tell the student to decompose to water and oxygen molecule. They will see that there are not enough oxygens present to do this, so they must double the initial amount of hydrogen peroxide. This is balancing an equation. Alternatively, this can be done in groups using models like fruit and toothpicks.
\end{itemize}

\subsubsection{Hard Water}
\begin{itemize}
\item Give students two tubes/bottles/beakers with water. One should be normal water and the other one hard water containing magnesium sulphate (Epsom salts). Have them determine which water is hard and which water is soft by adding the soap and shaking.
\item Ask students to wash their hands with soap (not
detergent) in hard and soft water.  In separate water bottles mix hard
water+soap, soft water + soap then shake.  Use a third bottle with
boiled temporary hard water to show temporary hardness.
\end{itemize}

\subsubsection{Volumetric Analysis}
\begin{itemize}
\item	In introduction to titration put some pictures of ``husbands/wives'' just outside the classroom door. Tell the class that there are some people who have come to find wife's but you don't remember how many of them there are. Send the girls out one-by-one until one comes back because there is no husband for her. Ask the class how many husbands were outside. This is titration.
You could also add a variation that one wife (acid) can have two husbands (bases) to demonstrate a titration with sulphuric acid.
\end{itemize}


%===================================================================================================
\section{Physics}
\subsection{Form I}
\subsubsection{Introduction to Physics}
\begin{itemize}
\item	Go outside. Walk around and ask students to point out different things that they think are related to physics. Since everything is fairly easily relatable to physics as long, as the students participate, it is a good way to show how applicable physics is in the real world. 
\item	And just be enthusiastic... its contagious. 
\end{itemize}

\subsubsection{Introduction to Laboratory Practice}
\begin{itemize}
\item	Put the class into groups. Give each group a piece of paper with different safety issues in the laboratory written on it.  Have each group put together a skit or song to relay the message to the rest of the class. 
\end{itemize}

\subsubsection{Measurement}
\begin{itemize}
\item	Rotating stations. Each station with question or two (measure the volume of this box, the length of this pen, which bottle has a larger volume, which rock has a larger mass... etc) If you don't have any measuring equipment, make some of your own...a ruler can be made with paper and pen, half of a water bottle as a beaker etc. 
\item	Play jeopardy with conversion questions. If the students cannot convert from km to mm it will be a problem for all physics Forms. One of the most important parts about physics is understanding unit conversion..so make it fun
\end{itemize}

\subsubsection{Force}
\begin{itemize}
\item	Have students stand in a very tight circle front to back. At the same time instruct them all to sit down. If they do it correctly they should be able to each sit on the lap of the student behind them. Its hilarious to watch, and there are forces all over the place. Like many of my suggestions however, more enjoyable than educational. 
\end{itemize}

\subsubsection{Archimedes Principle and Law of Flotation}
\begin{itemize}
\item	For flotation... you can have the students each bring in something they believe will float. Or if just bring in a variety of objects with different densities and have the students hypothesize which they think will float. Then have the students test each. 
\item	Have students form groups. Tell the students to try and build a boat (or just something that floats) and whichever groups boat can support the most weight is rewarded. 
\end{itemize}

\subsubsection{Structures and Properties of Matter}
\begin{itemize}
\item	Have students make a circle, maybe 10 feet in diameter. First show solid by filling the circle with blindfolded students. Tell them every time they touch another student they are to change direction. Since they are tightly packed in a solid, discuss the movement as vibration. Next remove most of the students. Leave, maybe 10 inside. They will be able to move more freely but will still interact. Lastly leave only two or three students to show a gas.  
\end{itemize}

\subsubsection{Pressure}
\begin{itemize}
\item	Use everyday situations. For example, the reason people sometimes put a khanga on their head when they carry a bucket full of water is to decrease the pressure on the top of their head. 
\item	I tried to do a bucket-with-water-on-head relay race using like 15 different teams. This taught them all very little about pressure, but from time to time having a bit of fun will do wonders in the classroom
\end{itemize}

\subsubsection{Work, Energy, and Power}
\begin{itemize}
\item	Put students into groups. Each group will have an object of known mass and they will be required to move that object a certain distance. I set up the activity so that each group would do the same amount of Work (those will small objects would have to go a larger distance, heavy objects not as far). And also ask them to record the amount of time it took to move the object. Then have the groups calculate and compare Work done and Power. 
\end{itemize}

\subsection{Form II}
\subsubsection{Static Electricity}
\begin{itemize}
\item	Visual Demonstration: Take a plastic bag with no holes in it, fill it with air, and close your fist around it making it into a make shift balloon. Rub the bag on your arm or head to electrify the bag.  Use the electrified bag to pick up small scraps of paper you have sprinkled on the table in front of you. 
\item	Static electricity can be difficult for students since the forces are invisible to the naked eye. The rubbing of the bag shows how a body can be electrified by ``rubbing'' ... and the paper scraps jumping onto the bag shows ``induction''
\end{itemize}

\subsubsection{Current Electricity}
\begin{itemize}
\item	Teacher draws a circuit on a piece of paper and then cuts it up to make a puzzle.  Students have to rearrange the pieces to make a circuit.
\item	To build off this...flash lights are easy to find... take a flash light apart, and use the labels used in the previous example. Have the students match the labels to the actual parts because it is very helpful to see what each piece looks like.
\end{itemize}

\subsubsection{Magnetism}
\begin{itemize}
\item	A compass can be made using the bottom of a water bottle, and floating a magnetized needle in the water. This tool can be used to show north and south and if you can find a magnet, you can use it to show the direction of magnetic field lines. 
\end{itemize}

\subsubsection{Forces in Equilibrium}
\begin{itemize}
\item	Find your own mass... see if there are any students who know how much they weigh. If not make the demonstration ``find the mass of a student''. Using a teeter-totter (easily made with a strong piece of wood or metal and something for it to teeter on) find the position of equilibrium between you and the student.  Since W1X1 = W2X2, and both distances (X1 and X2) are easily measured, you can find the unknown weight. Hooray for demonstrations
\end{itemize}

\subsubsection{Simple Mechanics}
\begin{itemize}
\item	Find a sturdy piece of timber (sturdy enough to support the weight of a student) and elevate one side of it. Have the student sit close to the elevated side and have the other students feel how easy it is to move him (teaches mechanical advantage). Then put the student of the far end of the stick and have the other students try to lift him putting effort in the middle of the piece of wood. This does a good job of showing 2nd and 3rd class levers.
\item	For 1st class, make a teeter-totter... be creative, you can do it. Show that if the load arm is short and the effort arm is long, you can support a lot more than just your own weight.
\end{itemize}

\subsubsection{Motion in Straight Line}
\begin{itemize}
\item	Another one that is more fun than educational... but that's because I like fun. Take the students to a big open area and have some kind of race (I went with wheelbarrow races and crab walk races) and have those not racing measuring the racers' velocities.
\end{itemize}

\subsubsection{Newton's Laws of Motion}
\begin{itemize}
\item	For Newton's first law: use a bottle, playing card and coin.  Flick the card from between the coin and bottle.  The coin doesn't move because there is no force applied to the coin.  
\end{itemize}

\subsubsection{Sustainable Energy Sources}
\begin{itemize}
\item	This is a tough topic for Form II since they will not learn about generators until Form IV.  But if you construct a small fan or a water wheel (again be creative, both are very doable with simple African resources) use them to show how wind and moving water can turn the wheels. From this motion it is easier I have found for students to see the relationship between natural resources (wind or water) and energy. 
\item	Solar energy can be shown with two bottles containing water, one left in the sun, one kept in the shade. The two will have different temperatures after not too long.
\end{itemize}

\subsection{Form III}
 \subsubsection{Application of Vectors}
\begin{itemize}
\item	Vector problems require practice practice practice. So, rather than just sit and do problems on the board, play a game or have a competition. For example, put students into teams and ask 10 or so vector questions. Working in teams can help the students feel more comfortable which will help to build confidence. At the end collect all the solutions and give the winning team some sort of reward. 
\item	Or, have just two teams. Give one team a chance to solve a problem. Team two will then have a chance to challenge to correct anything they believe to be an error. Go back and forth giving points for correct answers. Make sure the same students do not answer every single question for this game.
\end{itemize}


 \subsubsection{Optical Instruments}
\begin{itemize}
\item	Have students make a microscope using  clear hard plastic with a small hole, a drop of water, and a light source.
\item	Spoons are great tools. They can show both convex and concave mirrors. 
\end{itemize}

 \subsubsection{Thermal Expansion}
\begin{itemize}
\item	Put a sealed bottle containing warm water in class. Ask students to predict what will happen to the shape of the bottle as the water and air cools. Notice the bottle contracts (this may take more than just 80 minutes)
\item	By a balloon and put it on a bottle, put bottle in hot water and observe its rapid inflation... if you have ice water balloon will inflate inside the bottle which is awesome too
\end{itemize}

 \subsubsection{Transfer of Thermal Energy}
\begin{itemize}
\item	Bring in your jiko... show how that even though the mkaa is not in direct contact with the pot that the water will still boil (and the boiling water is a good example of convection). Also if your room is enclosed the temperature in the room will inevitably go up as well. Lots of thermal energy transfer going on here. 
\end{itemize}

 
\subsection{Form IV}
 \subsubsection{Waves}
\begin{itemize}
\item	For transverse: show very simply using a string
\item	For longitudinal I like to bring like ten students to the front and line them up a little less than an arms length a part. Tell them to close their eyes and push the one at the back of the line and observe. If you don't feel comfortable pushing your students, no problem I am sure there are plenty of other ways... this one makes me chuckle though, and does produce an excellent longitudinal wave. 
\item	For sounds waves/echoes/reverberations etc... most of the buildings and rooms are made of concrete. I find these perfect to let students explore their lung capacity and make some noise. Educational and therapeutic. 
\end{itemize}

 \subsubsection{Radioactivity}
\begin{itemize}
\item	Have students be electrons, neutrons or protons and act out beta, gamma, and alpha decay. Make sure you are not just telling them what to do though, have them figure out which particles should come and go on by themselves. 
\end{itemize}

 \subsubsection{Elementary Astronomy}
\begin{itemize}
\item	Have students come up with their own pneumonic devices to remember the planets. Then go through all the ideas and vote on a victor. This technique can be used for remembering things in certain orders really well because the student comes up with their own, and then hears 30-60 other pneumonic devices all with the same letter orders. 
\item	Give each student a fact that relates to a planet (or moon). Stick the names of those planets or moons on the wall around the room and have students try to match their fact with their planet or moon. 
\end{itemize}

%===================================================================================================
\section{Mathematics}
\subsection{Form I}
\subsubsection{Numbers}
\begin{itemize}
\item	Salama Says. Students act out points, lines, and angles using their arms.  
\item	Multiplication flashcards. Many Form I students don't know them and they are a good way to open the year.  
\item	Draw a number line on the ground.  Hand students different numbers on card paper and have them place themselves on the number line.  Emphasis decimals and fractions.
\end{itemize}

\subsubsection{Fractions, Decimals, and Percentages}
\begin{itemize}
\item  Use an orange or tangerine slices to show fractions to the students.
\end{itemize}

\subsubsection{Measurements}
\begin{itemize}
\item Have students use string, yard sticks, or rules to measure each other and plot on a graph.
\end{itemize}

\subsubsection{Geometry}
\begin{itemize}
\item Geoboards are a great way to teach geometry.  Consult the Math Specific Manual for more information.
\end{itemize}

\subsection{Form III}
\subsubsection{Relations}
\begin{itemize}
\item	Use dice or dominoes for sets; use practical examples of sets: husband and wife, teacher and student, etc. 
\end{itemize}

\subsubsection{Functions}
\begin{itemize}
\item	Use coordinate chart drawn on seed bag for students to stick coordinate points; Use of tables is essential and sometimes the students need many points to understand graphing; use geoboard to demonstrate functions.
\end{itemize}

\subsubsection{Statistics}
\begin{itemize}
\item	use little data and demonstrate mean, median, mode;
\item	have students gather and graph ages of 20 students; then find mean, median, mode              
\end{itemize}

\subsubsection{Rates and Variations}
\begin{itemize}
\item	Demonstrate 2:3:5 so give blocks, sticks, etc in that ratio;  Use many practical examples of direct and inverse variations; have students give examples since this will show real comprehension
\end{itemize}

\subsubsection{Sequences and Series}
\begin{itemize}
\item	Group work on Sequence and series; list the numbers of sequence to prove sum, etc.  Translate the question into meaningful math phrases or equations like 4th term is 10+ 2nd term; etc.
\end{itemize}

\subsubsection{Circle Theorems}
\begin{itemize}
\item	Cut out circles for different theorems to show and demonstrate to students; also have students draw what the questions states, in this way the students translates words into the circle picture to solve the question
\end{itemize}

\subsubsection{Earth as a Sphere}
\begin{itemize}
\item	Brought 3 globes to class so students can see latitude and longitude; solve problems as latitude and longitude change
\end{itemize}

\subsubsection{Statistics}
\begin{itemize}
\item	Had students at board to do cash flow, trial balance and balance sheet; demonstrate the ins and outs of accounting or cash flow
\end{itemize}

\section{English}
See the English Teaching Manual
