\section{Importance of English}

The most obvious reason to teach English is that the medium of instruction
and examinations are English. Thus, if students know English they
will be able to excel in the classroom and pass their final exams.\\


However, there are other reasons why teaching English is important.
English is the international language. If our students know English,
they are more likely to find work, travel, develop a worldly view,
and increase their standard of living. Learning another language also
stimulates the brain and can encourage development in analytical thinking.
Learning the grammar of English also helps students understand the
grammar and intricacies of their own language.\\


Data from the 2010 Peace Corps impact survey shows that Tanzanians
perceive the most significant impact of Peace Corps Volunteers in
Tanzania is increasing English skills and confidence. As a teacher
your biggest contribution will be speaking English. You may find yourself
being the only teacher committed to speaking only English at your
school, prepare yourself by reading this chapter. There will be days
you forget the value of your English-speaking presence, re-read the
following to get back on track.


\section{Why not Swahili?}

Teaching in English is obvious for English teachers but subject teachers
in Tanzania sometimes think it is easier to teach in Swahili. To be
blunt, teaching in Swahli is selfish, arrogant and lazy. Your native
language is English so you are the best resource for English language
learners at your school.\\


When you speak Swahili in schools in Tanzania:
\begin{itemize}
\item you dismiss your students' and teachers' abilities of learning English. 
\item your brightest students miss opportunities to improve their English.
They are likely to become teachers and continue to speak Swahili in
the classroom.
\item students are shown/taught that learning English is not worthwhile.
\item students are distracted or confused by your broken Swahili. 
\end{itemize}
The bottom line: don't be selfish, arrogant or lazy.

Here are some points for convincing yourself or others to use English.

\begin{tabular}{|>{\raggedright}p{5cm}||>{\raggedright}p{8cm}|}
\hline 
Excuse & Response\tabularnewline
\hline 
\hline 
My students do not know/understand English & Great, teach them!\tabularnewline
\hline 
\hline 
Swahili is easier to get the point across & Easy is not right, do not deny them the ability to learn English based
on your laziness to be creative or to figure out how to use level
appropriate English to teach your lessons\tabularnewline
\hline 
\hline 
My students prefer Swahili & Do they prefer working a jembe the rest of their lives? \tabularnewline
\hline 
\hline 
My students will not learn my subject (ICT, math, chemistry, biology,
physics), I canno't finish the syllabus without using Swahili & They learn most of the content for form I and II in primary school
(look at standard 6 and 7 textbooks); the point of form I and II is
learning English\tabularnewline
\hline 
\hline 
I need Swahili to teach lab safety & Two words: \textquotedbl{}stop\textquotedbl{} \textquotedbl{}go\textquotedbl{}\tabularnewline
\hline 
\hline 
I cannot build rapport with students as a mentor or counselor and
need Swahili to connect with them on that level & Use non-verbal communication. Or English only at school and be available
in non-academic setting with Swahili for mentoring.\tabularnewline
\hline 
\end{tabular}


\section{Introducing and maintaining a positive English learning environment}

Using results of the assessment below, brainstorm with students, teachers
and headmaster to make and enforce an English only environment. Students
all say they want to learn English so talk to them informally and
at assembly about how they can learn through speaking only English
at school (even at home!). If you involve students in the process
of making English the only language used in school, they will have
a sense of owning the English program and they will hold themselves
accountable. Be selective in inviting trouble makers, activists and
popular kids to make English cool. With administrators, students,
and small steps into an action plan for English medium and share with
staff and students.
\begin{itemize}
\item Start an English only policy or make a contract with the students
for teachers and students to sign and follow. Include what happens
when a student breaks the contract: written essays; letter apologizing
to the school and read aloud at assembly; song, speech or answering
questions in English at assembly.
\item Bring a map of the world to class and discuss languages emphasizing
the global use of English. Where is Kiswahili spoken/understood? What
language do they speak in (country)? Where do they speak (language)? 
\item Encourage teachers and students to attend school-wide activities:
debate, morning speech, spelling-bee, English club, etc.
\item Work with your counterpart to convince teachers and administration
of the importance of English. Use the 'Why not Swahili?' section (above)
in the staff room. Offer to help your teachers, especially English
teachers and form 6 leavers, improve their English.
\item Enforce language rules (positively) by giving prizes, movie nights,
stickers, baked goods to English speakers.
\item For Swahili speakers give detention as punishment and written English
essay to get out of detention. 
\item Give stickers to teachers to give to students who speak to them in
English. 
\item Inspire students to understand the importance of understanding versus
memorizing English. 
\item Teach your students note taking, test taking and study skills that
will improve their English on their own time.
\item Give each student a study journal. Collect their self assesments each
week. Give prompts such as: This week I learned; My strengths with
{[}the Periodic Table{]} are; I am still not understanding; I would
like help with; My learning and practicing plans for next week are;
This week I spoke English with:
\end{itemize}
\begin{tabular}{|c|c|c|}
\hline 
Name & Topic (weather, news, subject topic) & Amount of Time\tabularnewline
\hline 
\hline 
 &  & \tabularnewline
\hline 
\end{tabular}



\section{Assessment of English use in your school}

Invite teachers, administration, and students to join you in assessing
the English use in your school.
\begin{enumerate}
\item Who speaks English?


\textbigcircle{} Head of School


\textbigcircle{} Administrators


\textbigcircle{} Teachers


\textbigcircle{} Student leaders


\textbigcircle{} Students


\textbigcircle{} Staff


\textbigcircle{} Parents


\textbigcircle{} Other \_\_\_\_\_\_\_\_\_\_\_\_\_\_\_\_\_\_\_\_

\item What can you do to encourage English use?\\[60pt]
\item When/where is English used?


\textbigcircle{} Students announcing at parade


\textbigcircle{} Teachers and administrators announcing at parade


\textbigcircle{} In class with teacher


\textbigcircle{} In class without teacher


\textbigcircle{} In staff room


\textbigcircle{} At staff meetings


\textbigcircle{} At board meetings


\textbigcircle{} At student leader meetings


\textbigcircle{} At chai/during breaks


\textbigcircle{} On campus when school is in session


\textbigcircle{} On campus after school


\textbigcircle{} At soccer/netball fields

\item How can you make students and teachers accountable for English uses
at those times and venues? \\[60pt]
\item What English teaching or learning resources are available?


\textbigcircle{} Library


\textbigcircle{} Textbooks


\textbigcircle{} Fema magazines


\textbigcircle{} Other magazines


\textbigcircle{} Newspaper


\textbigcircle{} Picture cards


\textbigcircle{} Story cards


\textbigcircle{} Fluent English speakers (school staff, villagers,
expats)


\textbigcircle{} Internet


\textbigcircle{} Radio


\textbigcircle{} Videos/DVDs


\textbigcircle{} Video/DVD player


\textbigcircle{} Tapes/CDs


\textbigcircle{} Tape/CD player

\item Who has access to them? How can the resources available be put to
better use? \\[60pt]
\end{enumerate}

\section{English in the Classroom}


\subsection{Teaching English}
\begin{itemize}
\item Show students how to create bubble maps connecting all the concepts
they learned in one topic or subtopic. 
\item Create a vocabulary lists for each topic/subtopic.  
\item Teach peer editing and encourage students to edit each others notes,
essays, news articles and other writing. 
\item Write important words on one side of the board in simple English or
write Swahili  
\item Spell a long word (ex. Extraction) that your students should know.
How many words can they find using only the letters of that word (ex.
Cat, reaction, taxi, etc.)? 
\item Nouns in a bag (charades, pictionary, taboo) write concepts on small
paper and put them in a bag. Students select a piece of paper and
act, draw or describe the word without saying what is on the paper.
Example: lab safety/procedure/apparatus or first aid. 
\item Pronounce new words as you write the syllables. 
\item Rewrite simple exam questions with words they wont know to show them
how sad it would be if they missed the question because they don't
know the English. 
\item Train your students to organize their English learning and practice
regularly.  
\item Review the parts of speech - article, verb, noun, etc. Make fill-in-the
blank multiple choice with science or math words they do not know.
Does the blank sound like a verb, noun, adjective? Which part of speech
does each choice sound like?  
\item Make a list of common prefixes and suffixes. When new vocabulary has
prefixes or suffixes, ask students to define parts of words they know
before writing the definition.  
\item Relate words which are opposites or binary (multicellular/unicellular,
parallel/perpendicular, cation/anion). Have students say the opposite
whenever you say 'opposite' and point to a word. 
\end{itemize}

\subsection{Assessment}
\begin{itemize}
\item Test students collective knowledge with a group test. Write one question
on the board at a time and select a student at random to answer it.
All students get the collective grade. To avoid cheating, make sure
the class is silent and have the student come to the board before
you write the question. 
\item Spread out the capable students and assign group work. Give the whole
group the same grade or let them grade each other. 
\item Have students underline the words they don't know on assignments so
you will know what to teach or which words to use on a test.
\item Have students write down everything they know about (topic)
\item Dictation - read a couple sentences related to a recent topic and
have students write them as you read aloud. Write any new vocabulary
in the dictation on the board. Read a couple times slowly enough for
students to write.
\end{itemize}

\section*{Test Vocabulary}

\includegraphics{\string"C:/Users/Owner/Desktop/NEW/General Teaching/\string"\string"Test Vocabulary English-Swahili\string"}


\subsection{Lecturing in English}
\begin{itemize}
\item Level appropriate English = slow + repetition 
\item Before teaching a topic or lesson, choose 5-10 words that students
need to know. For example, when teaching about circulation, give students
a core vocabulary list for blood, heart, circulate, pump, and tube.
Do not translate every word in the notes, but rather be sure students
add to their own English vocabulary and increase their understanding
of the subject material. 
\item Students take turns to spell words (can be vocabulary from science/math
class) while others translate, give definitions, opposites or examples. 
\item Repeat instructions and commands so students understand.  
\item Use fill-in-the-blank notes for students to fill in as you are speaking.
 
\item Check for understanding by thumbs up/down, raise your hand if you
you understand. If students say they understand, ask one students
to explain it again to the class in English.  
\item Ask specific questions or why to make sure students understand.  
\item For difficult concepts, team teach - invite another teacher to teach
the topic with you so you are covering all angles.  
\item Team teaching is possible using a student or two from form 3 or 4.
It helps them learn by teaching and challenges them to speak English.
 
\item Leave an exercise book with the monitor/monitress for students to
anonymously write questions and list unknown words. Check these books
every afternoon. 
\item Use pictures and examples. If you are teaching about classification,
draw the organisms and quiz students. Common accidents in the chem-
istry lab could also be drawn and used to teach students the hazards
and new English vocabulary. Teach students shapes by drawing them
rather than their Kiswahili name. 
\item Draw or have students draw examples on pieces of paper and write what
they are on other pieces paper. Handout all of the papers at the beginning
of class and have students find the matching drawing or description,
then come to the front and describe the example. 
\item Have students create projects. By re-reading, summarizing, and presenting
from lessons, students reinforce what they have learned and increase
their confidence in writing and speaking English.  
\item Ask students to bring or make the teaching aids. 
\item Give out pieces of paper explaining a process and have students put
them in order (English NECTAs have a section on putting sentences
in order) 
\end{itemize}
\begin{center}
\setlength{\fboxsep}{0pt} \setlength{\fboxrule}{2pt} \fbox{} 
\par\end{center}


\section{Get Students Speaking English}
\begin{itemize}
\item Give students participation points for speaking in class 
\item \textbf{Kitimoto:} one student sits in front while others ask questions
to review subject material 
\item \textbf{Morning speech:} randomly invite a student or pair of students
to the front of the room to give a speech or ask each other questions
about a specific topic. Other students ask questions or challenge
the points made by speaker. 
\item \textbf{Question ball:} toss a ball or stuffed animal around the room;
thrower (or teacher) asks a question, catcher answers and asks another
question or asks for help (or ball returns to teacher) 
\item \textbf{20 questions:} try to guess a noun or concept by asking 20
or fewer yes or no questions. 
\item Post a schedule for students to sign up and present on a particular
topic or question.  
\item Get students speaking in front of class 5-10 minutes before you teach 
\item If you are talking about hearing, ask students to touch their ears.
When describing a lever, move your arm. If you are discussing temperature
regulation, have students act like they are in a hot/cold environment.
Teach form IV physics to do the wave.
\item Give students scenarios and have them make skits for topics like first
aid, fighting, lab safety, gravity/no gravity, perimeters/area or
ratios. 
\end{itemize}

\section{Out of Class English Activities}
\begin{itemize}
\item Book club - students read a book a month and discuss (prepare questions
in advance)  
\item Reading incentive - encourage students to read by a sticker chart
or clothes-pinning each name to a string that goes around the room
(up high so sneaking their clothes pin along the line creates a distraction)
move a certain distance for each book read or give the books points
by length and diculty. Award prizes for most points or books read
at the end of each month/term. 
\item Wikipedia has a simple English website: simple.wikipedia.org  
\item Students do writing exercises and read each others work. 
\item Write short stories for students to read 
\item Make books of familiar children's stories or your own.  
\item Reading hour - invite students to come with a book (bring books, magazines
and newspapers if you have them, and bring your dictionaries be around
to explain if they don't understand what they're reading).  
\item FEMA publishes \textit{Fema Magazine User's Guide } for how to use
Fema magazines that have great lesson plans for life skills, reading,
writing, debate, discussion, etc. 
\item Listen to popular songs and write the lyrics.  
\item Books on tape can be downloaded from Librivox.org or similar sites. 
\item Read aloud from a book, newspaper, students' writing or your own. 
\item Make informal English time by asking students questions about what
they're doing or what they learned after greeting them, on breaks,
after school or in the village. 
\item English conversation group - PSKK (or Piga Story Kwa Kiingereza) Club
- Format can be very flexible; meeting bi-weekly, students are presented
with a topic or theme of discussion which may be decided in advance
so they can prepare (see writing/discussion ideas). 
\item Interviews - students to interview each other or members of the community
who speak English.  
\item Read aloud - students to read aloud from subject material, story books
or stories they have written.  
\item If you coach sports, enforce your schools English policy on the field,
track or court. 
\item School newspaper - make a template for student journalists to fill
in with weekly articles, editorials, advertisements, and comics. (Alternative:
science journal)  
\item Literary zine - pay to publish (i.e. printed and bound in a stationery
or printed and stapled) biannual collection of students' fiction,
non-fiction, poetry with pictures. In some regions it might be cheaper
to have this printed in and sent from the States.)  
\item If you have a computer lab with internet access, a personal, class
or school blog project might work; or set up a private social network
on a site like www.xanga.com 
\item World Wise media exchange - type up your students letters in an email
with pictures, make video clips of your students making ugali, in
class, doing cleanliness, etc. Have students draw pictures, write
stories, or make short movies and invite the American class to react
with a story, poem, movie, drawing, photo or song. 
\item Movies - show movies with subtitles (downloadable at www.subscene.net).
Planet Earth and Life are great series for pace of English but speakers
aren't visible. If dialogue is too fast, you can slow a movie down
with VLC media player.
\item Group story - write the introductions of several stories (a sentence
or paragraph about anything) on different pieces of paper paper. Have
students pass the papers around adding one sentence to each story. 
\item Write a story on the board and edit it using coloured chalk and editing
symbols. 
\item Comic strips - students love the photo comic section of FEMA public-
cations, introduce them to drawn comics with storylines.  
\item Music free-style rapping/church choir/school welcome party song and
dance in English 
\item Keep a class journal (or several) for students to write stories or
questions. Encourage students to answer each others questions or comment
on each others stories. If they're short on ideas, write a prompt
every couple pages or bind envelopes together and write a prompt on
each one, have students write answers on their own paper and put them
in the envelopes.  
\item Letter writing - if you don't have pen pals in the USA, try being
pen pals with a neighbouring school. Students will be on the same
language level, have better chances of meeting each other, and won't
have any reasons to beg. 
\item Essay contest - descriptive or analytical writing about a prompt.
A good enough prize will get the whole school writing. Beware: you
might get tired of reading the same thing.  
\item Spelling Bees! Students take turns spelling words with increasing
difficulty. They are out if they misspell a word. Each week a different
stream competes and winners are selected from each. Then have the
winners from each stream compete for the school title.  
\item School debate between forms. Let the students choose the motion and
be the judge, being sure they score students on points, explanation,
grammar, pronunciation, and confidence.  
\item Make a weekly contest, one for stories, another week for songs, another
week for poems or comics, throughout the year. Then compile the best
into a literary magazine for the students at the end of the semester
or year. 
\end{itemize}

\section{Games}
\begin{itemize}
\item Speed scrabble (aka Bananagrams) Start with all letters face down.
Players start with 7 letters each and work independently to spell
connected (like in scrabble or on a crossword puzzle) words. When
a player has used up all their letters, they say 'go' and everyone
picks another letter. Letters are rearranged to include the new letter.
Game continues until all letters are used. Winner is the first to
use all his or her letters (or the one who says `go' when there are
no letters remaining). Everyone reads their words aloud. Paint the
bottle caps or use the same soda to avoid cheating.  
\item Silly sentences - Write phrases (verb, subject, time, and prepositional,
see examples below) on cards and draw symbols (or use matching stickers)
on the back of each card for each type of phrase. With the cards symbol
side up have students select one of each and make a sentence, conjugating
the verb to match the time. Have one student write the sentence, another
translate, and a third draw a picture.: 

\begin{itemize}
\item \textbf{time phrases}: occasionally, yesterday, this afternoon, last
night 
\item \textbf{subject phrases}: my (brother/sister), the headmaster, a long,
poisonous, hungry snake named Juma  
\item \textbf{verb phrases}: (wash) my clothes, (eat) rice and beans, (wear)
a purple dress, (get) pregnant  
\item \textbf{prepositional phrases}: on a deserted island, at a birthday
party, on TV 
\end{itemize}

A sentence might be: Last night the headmaster washed my clothes at
a birthday party. 

\end{itemize}

\section{Debate topics}
\begin{itemize}
\item Globalization 
\item Use nightly news for topics (kipimo joto)  
\item Based on local village issues  
\item Tanzania is being taken over by the free masons  
\item Loliondo medicine man 
\item Western health vs. medicine  
\item African nations will never develop without foreign aid  
\item Arsenal vs. Mann U  
\item Tanzania vs. Kenya  
\item Bongo flava is better than rap  
\item Tanzanian education should be in Kiswahili  
\item A woman should always cook  
\item Students should wear uniforms to school  
\item Birth control is a woman's responsibility  
\item Corporal punishment  
\item Day vs. boarding  
\item Education is better than money  
\item Private schools are better than government schools  
\item HIV will never be eradicated  
\item Students results are because of teachers  
\item English should be the medium for primary education 
\end{itemize}

\section{Writing/Discussion Prompts}
\begin{itemize}
\item Auto/biography 
\item Describe people, places, hobbies, interests, culture, and custom 
\item Myths and fables  
\item How to  
\item Poetry and song lyrics  
\item Hot topic editorials (abortion, witch doctor)  
\item Book review  
\item Write or talk about a photograph or item of sentimental value  
\item Interview with a family member, friend or local professional  
\item Religion  
\item Philosophy  
\item What or whom inspires you to be a better student?  
\item Draw a family tree and tell your family history or describe the people
in your family. 
\item Where have you been? What made a particular trip or school break memorable? 
\item FEMA articles  
\item Current events \end{itemize}


